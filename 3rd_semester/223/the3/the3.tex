\documentclass[12pt]{article}
\usepackage[utf8]{inputenc}
\usepackage[dvips]{graphicx}
\usepackage{epsfig}
\usepackage{fancybox}
\usepackage{verbatim}
\usepackage{array}
\usepackage{latexsym}
\usepackage{alltt}
\usepackage{float}
\usepackage{amsmath}
\usepackage{hyperref}
\usepackage{listings}
\usepackage{color}
\usepackage[hmargin=3cm,vmargin=5.0cm]{geometry}
\topmargin=-1.8cm
\addtolength{\textheight}{6.5cm}
\addtolength{\textwidth}{2.0cm}
\setlength{\oddsidemargin}{0.0cm}
\setlength{\evensidemargin}{0.0cm}

\newcommand{\HRule}{\rule{\linewidth}{1mm}}
\newcommand{\kutu}[2]{\framebox[#1mm]{\rule[-2mm]{0mm}{#2mm}}}
\newcommand{\gap}{ \\[1mm] }

\newcommand{\Q}{\raisebox{1.7pt}{$\scriptstyle\bigcirc$}}

\lstset{
    %backgroundcolor=\color{lbcolor},
    tabsize=2,
    language=C++,
    basicstyle=\footnotesize,
    numberstyle=\footnotesize,
    aboveskip={0.0\baselineskip},
    belowskip={0.0\baselineskip},
    columns=fixed,
    showstringspaces=false,
    breaklines=true,
    prebreak=\raisebox{0ex}[0ex][0ex]{\ensuremath{\hookleftarrow}},
    %frame=single,
    showtabs=false,
    showspaces=false,
    showstringspaces=false,
    identifierstyle=\ttfamily,
    keywordstyle=\color[rgb]{0,0,1},
    commentstyle=\color[rgb]{0.133,0.545,0.133},
    stringstyle=\color[rgb]{0.627,0.126,0.941},
}


\begin{document}



\noindent
\HRule \\[3mm]
\small
\begin{tabular}[b]{lp{3.8cm}r}
{} Middle East Technical University &  &
{} Department of Computer Engineering \\
\end{tabular} \\
\begin{center}

                 \LARGE \textbf{CENG 223} \\[4mm]
                 \Large Discrete Computational Structures \\[4mm]
                \normalsize Fall '2020-2021 \\
                    \Large Homework 3 \\
                \normalsize Student Name and Surname: Mustafa Sezgin  \\
                \normalsize Student Number: 2380863 \\
\end{center}
\HRule


\section*{Question 1}
Equivalent to $(2^{22} + 4^{44} + 6^{66} + 8^{80} + 10^{110}) \bmod 11$, we have
\begin{equation}
    ((2^{22} \bmod 11) + (4^{44} \bmod 11) + (6^{66} \bmod 11) + (8^{80} \bmod 11) + (10^{110} \bmod 11)) \bmod 11 \label{eq:1}
\end{equation}
using \textit{Corollary 2} from Chapter 4.1 in the textbook. Also, since 11 is a prime number, and 2 is not divisible by 11, then $2^{10} \bmod 11 \equiv 1$ by \textit{Fermat's Little Theorem}. Thus,
\begin{align*}
    2^{22} \bmod 11 & \equiv 2^2 (2^{10})^{2} \bmod 11 & \\
    & \equiv 2^2 (1)^2 \bmod 11 & \textit{by Fermat's Little Theorem} \\
    & \equiv 4 \bmod 11 & \\
    & \equiv 4, & 
\end{align*}
\begin{align*}
    4^{44} \bmod 11 & \equiv 2^{88} \bmod 11 & \\
    & \equiv 2^8 (2^{10})^8 \bmod 11 & \\
    & \equiv 2^8 (1)^2 \bmod 11 & \textit{by Fermat's Little Theorem} \\
    & \equiv 256 \bmod 11 & \\
    & \equiv 3, & 
\end{align*}
\begin{align*}
    6^{66} \bmod 11 & \equiv (2^{66}) (3^{66}) \bmod 11 & \\
    & \equiv 2^{66} (-8)^{66} \bmod 11 & \textit{since $3 \equiv -8 (\bmod 11)$} \\
    % & \equiv 2^{66} (8)^{66} \bmod 11 & \\
    & \equiv 2^{66} (-2^3)^{66} \bmod 11 & \\
    & \equiv 2^{264} \bmod 11 & \\
    & \equiv 2^4 (2^{10})^{26} \bmod 11 & \\
    & \equiv 2^4 (1)^{26} \bmod 11 & \textit{by Fermat's Little Theorem} \\
    & \equiv 16 \bmod 11 & \\
    & \equiv 5, & 
\end{align*}
\begin{align*}
    8^{80} \bmod 11 & \equiv (2^3)^{80} \bmod 11 & \\
    & \equiv 2^{240} \bmod 11 & \\
    & \equiv (2^{10})^{24} \bmod 11 & \\
    & \equiv (1)^{24} \bmod 11 & \textit{by Fermat's Little Theorem} \\
    & \equiv 1 \bmod 11 & \\
    & \equiv 1. & 
\end{align*}
And finally,
\begin{align*}
    10^{110} \bmod 11 & \equiv (-1)^{110} \bmod 11 & \textit{since $10 \equiv -1 (\bmod 11)$} \\
    & \equiv 1 \bmod 11 & \\
    & \equiv 1. & 
\end{align*}
Substituting the results into the equation \eqref{eq:1}, we get $(4 + 3 + 5 + 1 + 1) \bmod 11 \equiv 14 \bmod 11 \equiv 3$.

\section*{Question 2}
We will use \textit{Euclidean Algorithm} to calculate $\gcd{(5n + 3, 7n + 4)}$. We know that $5n + 3 < 7n + 4$, $\forall n > 0$, and that
\[
    7n + 4 = 1 (5n + 3) + (2n + 1),
\]
where $0 \le 2n + 1 < 5n + 3$. Any common divisor of $7n + 4$ and $5n + 3$ must also be a divisor of $2n + 1$, then $\gcd{(7n + 4, 5n + 3)} = \gcd{(5n + 3, 2n + 1)}$.
\[
    5n + 3 = 2 (2n + 1) + (n + 1),
\]
where $0 \le n + 1 < 2n + 1$. Any common divisor of $5n + 3$ and $2n + 1$ must also be a divisor of $n + 1$, then $\gcd{(5n + 3, 2n + 1)} = \gcd{(2n + 1, n + 1)}$.
\[
    2n + 1 = 1 (n + 1) + (n),
\]
where $0 \le n < n + 1$. Any common divisor of $2n + 1$ and $n + 1$ must also be a divisor of $n$, then $\gcd{(2n + 1, n + 1)} = \gcd{(n + 1, n)}$. If $n = 1$, then
\[
    n + 1 = 2 = 2 (1) + (0) = 2 (n) + (0),
\]
and $\gcd{(n + 1, n)} = \gcd{(2, 1)} = 1$, hence $\gcd{(5n + 3, 7n + 4)} = \gcd{(8, 11)} = 1$. If $n > 1$, then
\[
    n + 1 = 1 (n) + (1),
\]
where $0 \le 1 < n$. Any common divisor of $n + 1$ and $n$ must also be a divisor of $1$, then $\gcd{(n + 1, n)} = \gcd{(n, 1)}$.
\[
    n = n (1) + (0),
\]
where $0 \le 0 < 1$ and $1$ is a divisor of $n$. Thus, $\gcd{(5n + 3, 7n + 4)} = 1$. \\

\noindent If $n = 0$, then $\gcd{(5n + 3, 7n + 4)} = \gcd{(3, 4)} = 1$ since $3$ and $4$ are relatively prime. \\
All in all, we have proved that $\gcd{(5n + 3, 7n + 4)} = 1$, $\forall n \ge 0$.

\newpage
\section*{Question 3}
For a prime number $x$, and integers $m$, $n$, and $k$, we are given that
\begin{align*}
    m^2 = n^2 + k x \\
    m^2 - n^2 = k x.
\end{align*}
Since $k$ is an integer,
\begin{align*}
    x &\mid (m^2 - n^2) \\
    x &\mid (m + n)(m - n).
\end{align*}
Since $x$ is a prime number, then $x \mid (m + n) \ \vee \ x \mid (m - n)$ by \textit{Lemma 3} from Chapter 4.3 in the textbook.


\section*{Question 4}
$\forall n \ge 1$, let $P(n)$ be the predicate for $1 + 4 + 7 + \dots + (3n - 2) = \frac{n (3n - 1)}{2}$. \\ \\
\textit{Base Step:} When $n = 1$, $P(1)$ is true since
\begin{align*}
    1 &= \frac{1 (3 \cdot 1 - 1)}{2} \\
    1 &= \frac{1 \cdot 2}{2} \\
    1 &= 1.
\end{align*}
\textit{Inductive Step:} Assume $P(k)$ is true for some $k \ge 1$, that is,
\[
    1 + 4 + 7 + \dots + (3k - 2) = \frac{k (3k - 1)}{2}.
\]
Then
\begin{align*}
    1 + 4 + 7 + \dots + (3k - 2) + (3(k + 1) - 2) &= \frac{k (3k - 1)}{2} + 3(k + 1) - 2 \\
    &= \frac{3k^2 - k}{2} + 3k + 1 \\
    &= \frac{3k^2 - k}{2} + \frac{6k + 2}{2} \\
    &= \frac{3k^2 + 5k + 2}{2} \\
    &= \frac{(k + 1)(3k + 2)}{2} \\
    &= \frac{(k + 1)(3(k + 1) - 1)}{2},
\end{align*}
which means $P(k + 1)$ is also true. By induction, $P(n)$ is true for all $n \ge 1$.

\end{document}

