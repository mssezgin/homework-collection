\documentclass[11pt]{article}
\usepackage[utf8]{inputenc}
\usepackage{float}
\usepackage{amsmath}
\usepackage{amssymb}

\usepackage[hmargin=3cm,vmargin=6.0cm]{geometry}
%\topmargin=0cm
\topmargin=-2cm
\addtolength{\textheight}{6.5cm}
\addtolength{\textwidth}{2.0cm}
%\setlength{\leftmargin}{-5cm}
\setlength{\oddsidemargin}{0.0cm}
\setlength{\evensidemargin}{0.0cm}

% symbol commands for the curious
\newcommand{\setZp}{\mathbb{Z}^+}
\newcommand{\setR}{\mathbb{R}}
\newcommand{\calT}{\mathcal{T}}

\begin{document}

\section*{Student Information } 
%Write your full name and id number between the colon and newline
%Put one empty space character after colon and before newline
Full Name : Mustafa Sezgin \\
Id Number : 2380863 \\

% Write your answers below the section tags
\section*{Answer 1}
We have three placeholders $S$, $H$, and $N$ corresponding to stars, habitable planets, and non-habitable planets respectively. There are three cases for our universe: between the two habitable planets, there can be six, seven or eight non-habitable planets. From innermost (left) to outermost (right); \\ \\
\textit{Case 1:}
\[ S, [H, N, N, N, N, N, N, H], [N], [N] \]
Three groups of planets can be sorted in $\frac{3!}{2!} = 3$ different ways, since two of them are identical. \\ \\
\textit{Case 2:}
\[ S, [H, N, N, N, N, N, N, N, H], [N] \]
Two groups of planets can be sorted in $2! = 2$ different ways. \\ \\
\textit{Case 3:}
\[ S, [H, N, N, N, N, N, N, N, N, H] \]
One group of planets can be sorted in $1! = 1$ different way. By basic sum rule of counting, in total, there are $3 + 2 + 1 = 6$ permutations for placeholders, and $P(10, 1) \cdot P(20, 2) \cdot P(80, 8)$ permutations to fill in them. By basic product rule of counting, there are
\[ 6 \cdot P(10, 1) \cdot P(20, 2) \cdot P(80, 8) = 6 \cdot 10 \cdot 20 \cdot 19 \cdot 80 \cdot 79 \cdot 78 \cdot 77 \cdot 76 \cdot 75 \cdot 74 \cdot 73 \]
different universes.

\section*{Answer 2}
\begin{align*}
    a_n = 2 a_{n-1} + 15 a_{n-2} - 36 a_{n-3} + 2^n \\
    a_n - 2 a_{n-1} - 15 a_{n-2} + 36 a_{n-3} = 2^n
\end{align*}
The characteristic equation is
\begin{align*}
    r^3 - 2r^2 - 15r + 36 = 0 \\
    (r - 3)^2 (r + 4) = 0 \\
    r_{1,2} = 3, \ r_3 = -4.
\end{align*}
Homogeneous solution is $a_n^{(h)} = c_1 3^n + c_2 n 3^n + c_3 (-4)^n$. \\
Particular solution is of the form $a_n^{(p)} = A 2^n$.
\begin{align*}
    A 2^n - 2 A 2^{n - 1} - 15 A 2^{n - 2} + 36 A 2^{(n - 3)} &= 2^n \\
    A 2^3 - 2 A 2^2 - 15 A 2^1 + 36 A &= 2^3 \\
    6A &= 8 \\
    A &= \frac{4}{3}
\end{align*}
Then
\begin{align*}
    a_n &= a_n^{(h)} + a_n^{(p)} \\
    &= c_1 3^n + c_2 n 3^n + c_3 (-4)^n + \frac{4}{3} 2^n
\end{align*}

\section*{Answer 3}
A valid $n$-digit activation code can be represented as $d_1 d_2 d_3 \dots d_{n - 1} d_n$, where $d_i$ is a decimal digit. There are two cases for an $n$-digit activation code: last digit of it can be even or odd. \\ \\
\textit{Case 1:} $d_n$ is even. Then $d_1 d_2 d_3 \dots d_{n - 1}$ has odd number of odd digits, which means it is a valid \textit{n-1}-digit activation code. In this case, by basic product rule of counting,
\begin{align*}
    \text{$\#$ of $n$-digit activation codes} &= \text{$\#$ of \textit{n-1}-digit activation codes $\times$ $\#$ of even $d_n$s} \\
    &= a_{n - 1} \times 5,
\end{align*}
since $d_1 d_2 d_3 \dots d_{n - 1}$ is a valid \textit{n-1}-digit activation code, and $d_n \in \{0, 2, 4, 6, 8\}$. \\ \\
\textit{Case 2:} $d_n$ is odd. Then $d_1 d_2 d_3 \dots d_{n - 1}$ has even number of odd digits, which means it is an invalid \textit{n-1}-digit activation code. In this case, by basic product rule of counting,
\begin{align*}
    \text{$\#$ of $n$-digit activation codes} &= \text{$\#$ of invalid \textit{n-1}-digit activation codes $\times$ $\#$ of odd $d_n$s} \\
    &= (10^{n - 1} - a_{n - 1}) \times 5,
\end{align*}
since $\#$ of invalid \textit{n-1}-digit activation codes = total $\#$ of \textit{n-1}-digit numbers $-$ $\#$ of valid \textit{n-1}-digit activation codes, and $d_n \in \{1, 3, 5, 7, 9\}$. \\ \\
By basic sum rule of counting, the number of valid $n$-digit activation codes is
\begin{align*}
    a_n &= 5 a_{n - 1} + 5 (10^{n - 1} - a_{n - 1}) \\
    &= 5 \cdot 10^{n - 1}.
\end{align*}
Furthermore, we have that
\begin{align*}
    a_1 &= 5 \cdot 10^{1 - 1} \\
    &= 5,
\end{align*}
and that
\begin{align*}
    a_{n - 1} &= 5 \cdot 10^{n - 2} \\
    10 a_{n - 1} &= 10 \cdot 5 \cdot 10^{n - 2} \\
    &= 5 \cdot 10^{n - 1} \\
    &= a_n.
\end{align*}
Thus, the recurrence relation for the number of valid $n$-digit activation codes is $a_n = 10 a_{n - 1}$ with the initial condition $a_1 = 5$.

\section*{Answer 4}
Let $A(x) = \sum_{k = 0}^\infty a_k x^k$ be the generating function for the sequence $\{a_k\}$. Using the recurrence relation, we have that
\begin{align*}
    A(x) &= \sum_{k = 0}^\infty a_k x^k \\
    &= a_0 + a_1 x + a_2 x^2 + \sum_{k = 3}^\infty a_k x^k \\
    &= a_0 + a_1 x + a_2 x^2 + \sum_{k = 3}^\infty (3 a_{k - 1} - 3 a_{k - 2} + a_{k - 3}) x^k \\
    &= a_0 + a_1 x + a_2 x^2 + 3 \sum_{k = 3}^\infty a_{k - 1} x^k - 3 \sum_{k = 3}^\infty a_{k - 2} x^k + \sum_{k = 3}^\infty a_{k - 3} x^k \\
    &= a_0 + a_1 x + a_2 x^2 + 3 \sum_{k = 2}^\infty a_k x^{k + 1} - 3 \sum_{k = 1}^\infty a_k x^{k + 2} + \sum_{k = 0}^\infty a_k x^{k + 3} \\
    &= a_0 + a_1 x + a_2 x^2 + 3x \sum_{k = 2}^\infty a_k x^k - 3x^2 \sum_{k = 1}^\infty a_k x^k + x^3 \sum_{k = 0}^\infty a_k x^k \\
    &= a_0 + a_1 x + a_2 x^2 + 3x (A(x) - a_0 - a_1 x) - 3x^2 (A(x) - a_0) + x^3 A(x) \\
    &= a_0 + a_1 x + a_2 x^2 + 3x A(x) - 3 a_0 x - 3 a_1 x^2 - 3x^2 A(x) + 3 a_0 x^2 + x^3 A(x).
\end{align*}
Then
\begin{align*}
    A(x) - 3x A(x) + 3x^2 A(x) - x^3 A(x) &= a_0 + a_1 x + a_2 x^2 - 3 a_0 x - 3 a_1 x^2 + 3 a_0 x^2 \\
    (1 - 3x + 3x^2 - x^3) A(x) &= a_0 + (a_1 - 3 a_0) x + (a_2 - 3 a_1 + 3 a_0) x^2 \\
    (1 - x)^3 A(x) &= 1 + (3 - 3 \cdot 1) x + (6 - 3 \cdot 3 + 3 \cdot 1) x^2 \\
    (1 - x)^3 A(x) &= 1,
\end{align*}
and the generating function can be simplified as the following
\begin{align*}
    A(x) &= \frac{1}{(1 - x)^3} \\
    &= \sum_{k = 0}^\infty C(3 + k - 1, 3 - 1) x^k \quad \text{(*)} \\
    &= \sum_{k = 0}^\infty C(k + 2, 2) x^k \\
    &= \sum_{k = 0}^\infty \frac{(k + 2)(k + 1)}{2} x^k \\
    &= \sum_{k = 0}^\infty a_k x^k,
\end{align*}
where line (*) is obtained using the ninth function in \textit{Table 1} from Chapter 8.4 in the textbook. Therefore, the solution of the recursive equation is $a_k = \frac{(k + 2)(k + 1)}{2}$.

\section*{Answer 5}
\paragraph{a.}
\begin{itemize}
    \item Since $m + n = n + m$ for all $m, n \in \mathbb{Z}^+$, then $((m, n), (m, n)) \in R$. Thus, $R$ is reflexive.
    \item For all $k, l, m, n \in \mathbb{Z}^+$, if $((k, l), (m, n)) \in R$, then $k + n = l + m$. Rearranging the equation, we have $m + l = n + k$, that is $((m, n), (k, l)) \in R$. Thus, $R$ is symmetric.
    \item For all $k, l, m, n, p, q \in \mathbb{Z}^+$, if $k - l = m - n$ and $m - n = p - q$, then $k - l = p - q$. In other words, if $k + n = l + m$ and $m + q = n + p$, then $k + q = l + p$. This means if $((k, l), (m, n)) \in R$ and $((m, n), (p, q)) \in R$, then $((k, l), (p, q)) \in R$. Thus, $R$ is transitive.
\end{itemize}
Since $R$ is reflexive, symmetric, and transitive, $R$ is an equivalence relation.

\paragraph{b.}
\[ [(1, 2)]_R = \{(m, n) \mid m, n \in \mathbb{Z}^+ \wedge ((1, 2), (m, n)) \in R\} \]
Since $1 + n = 2 + m$ for any $(m, n) \in [(1, 2)]_R$, by the definition of the equivalence relation $R$, then $n = m + 1$, and
\[ [(1, 2)]_R = \{(m, m + 1) \mid m \in \mathbb{Z}^+ \wedge ((1, 2), (m, m + 1)) \in R\}. \]

\end{document}