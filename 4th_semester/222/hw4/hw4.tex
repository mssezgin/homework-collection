\documentclass[12pt]{article}
\usepackage[utf8]{inputenc}
\usepackage{float}
\usepackage{amsmath}


\usepackage[hmargin=3cm,vmargin=6.0cm]{geometry}
\topmargin=-2cm
\addtolength{\textheight}{6.5cm}
\addtolength{\textwidth}{2.0cm}
\setlength{\oddsidemargin}{0.0cm}
\setlength{\evensidemargin}{0.0cm}
\usepackage{indentfirst}
\usepackage{amsfonts}
\usepackage{parskip}
%\usepackage[T1]{fontenc}
%\usepackage{palatino}

\begin{document}

\section*{Student Information}

Name : Mustafa Sezgin

ID : 2380863 \\


\textbf{a.} This Monte Carlo simulation with size $N = 21141$ guarantees with 98\% confidence that the probability that the total weight is greater than 640 tons is in the interval $\hat{p} \pm 0.008$. This size is calculated as
\begin{align*}
    N &\ge 0.25 \left( \frac{z_{0.01}}{0.008} \right)^2 \\
    N &\ge 21140.21
\end{align*}
Result of the simulation is $\hat{p} = 0.1259$; thus, $p \in [0.1179, 0.1339]$. \\

\textbf{b.} The expected value of the total weights is 599.0387 tons. \\ % Thus, we can expect the total weight of the plastics to not exceed 640 tons. \\ % The result of the simulation shows that the expected value of the total weight is $E(X) = 599.0387$. \\

\textbf{c.} The standard deviation of the total weights is 35.7356. \\

The random variable $W$ for the total weights has normal approximation. The probability that the total weight is greater than 640 tons is
\[
    P(W > 640) = P\left( \frac{W - 599.0387}{35.7356} > \frac{640 - 599.0387}{35.7356} \right) = 1 - \Phi(1.1462) = 0.1259.
\]
This is consistent with the findings in part a.

% The null hypothesis is $H_0: \mu = 640$, and the alternative hypothesis is $H_A: \mu > 640$. We obtained from this simulation that $\bar{X} = 599.0387$ and $\sigma(\bar{X}) = 35.7356$.
% \[
%     \frac{\bar{X} - \mu_0}{\sigma(\bar{X})} = \frac{599.0387 - 640}{35.7356} = -1.146232773
% \]
% 640 has a $Z$-value $-1.146232773$ in a normal distribution with $\mu = 599.0387$ and $\sigma = 35.7356$. This implies that the total weight exceeds 640 tons with probability $1 - \Phi(-1.146232773) = 1 - 0.125855 = 0.87415$, which is consistent with the findings in part a. \\ % The result of the simulation shows that the standard deviation $\sigma(X) = 35.7356$.


% octave:1> hw4
% Estimated probability = 0.130126
% Expected weight = 599.464502
% Standard deviation = 35.786116
% octave:2> hw4
% Estimated probability = 0.123693
% Expected weight = 599.037257
% Standard deviation = 35.598410
% octave:3> hw4
% Estimated probability = 0.124923
% Expected weight = 598.917299
% Standard deviation = 35.755980
% octave:4> hw4
% Estimated probability = 0.123220
% Expected weight = 598.614722
% Standard deviation = 35.643664
% octave:5> hw4
% Estimated probability = 0.124024
% Expected weight = 599.089480
% Standard deviation = 35.318577
% octave:6> hw4
% Estimated probability = 0.128187
% Expected weight = 599.149782
% Standard deviation = 35.994593
% octave:7> hw4
% Estimated probability = 0.127430
% Expected weight = 598.997766
% Standard deviation = 36.051843
% 
% 0.1259432857, 599.0386869, 35.73559757
% 
% --------------------------------------
% 
% N = 21140.2126213060
% 
% N = 21140
% octave:3> hw4
% Estimated probability = 0.130747
% Expected weight = 599.118230
% Standard deviation = 36.006134
% 
% N = 21141
% octave:1> hw4
% Estimated probability = 0.126295
% Expected weight = 599.112537
% Standard deviation = 35.878687
% octave:2> hw4
% Estimated probability = 0.125018
% Expected weight = 598.980901
% Standard deviation = 35.614077
% octave:3> hw4
% Estimated probability = 0.124308
% Expected weight = 599.198870
% Standard deviation = 35.605182
%
% 0.125207, 599.097436, 35.69931533
%
% N = 21141 opt
% octave:5> opthw4
% Estimated probability = 0.123741
% Expected weight = 599.028994
% Standard deviation = 35.617563
% octave:6> opthw4
% Estimated probability = 0.125112
% Expected weight = 599.163494
% Standard deviation = 35.435826
% octave:7> opthw4
% Estimated probability = 0.128660
% Expected weight = 599.072737
% Standard deviation = 36.030737
%
% 0.1258376667, 599.0884083, 35.69470867

\end{document}
