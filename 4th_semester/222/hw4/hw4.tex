\documentclass[12pt]{article}
\usepackage[utf8]{inputenc}
\usepackage{float}
\usepackage{amsmath}


\usepackage[hmargin=3cm,vmargin=6.0cm]{geometry}
\topmargin=-2cm
\addtolength{\textheight}{6.5cm}
\addtolength{\textwidth}{2.0cm}
\setlength{\oddsidemargin}{0.0cm}
\setlength{\evensidemargin}{0.0cm}
\usepackage{indentfirst}
\usepackage{amsfonts}
\usepackage{parskip}

\begin{document}

\section*{Student Information}

Name : Mustafa Sezgin

ID : 2380863 \\


\textbf{a.} This Monte Carlo simulation with size $N = 21141$ guarantees with 98\% confidence that the probability that the total weight is greater than 640 tons is in the interval $\hat{p} \pm 0.008$. This size is calculated as
\begin{align*}
    N &\ge 0.25 \left( \frac{z_{0.01}}{0.008} \right)^2 \\
    N &\ge 21140.21
\end{align*}
Result of the simulation is $\hat{p} = 0.1259$; thus, $p \in [0.1179, 0.1339]$. \\

\textbf{b.} The expected value of the total weights is 599.0387 tons. \\

\textbf{c.} The standard deviation of the total weights is 35.7356. \\

The random variable $W$ for the total weights has normal approximation. The probability that the total weight is greater than 640 tons is
\[
    P(W > 640) = P\left( \frac{W - 599.0387}{35.7356} > \frac{640 - 599.0387}{35.7356} \right) = 1 - \Phi(1.1462) = 0.1259.
\]
This is consistent with the findings in part a.

\end{document}
