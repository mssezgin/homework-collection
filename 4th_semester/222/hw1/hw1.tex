\documentclass[12pt]{article}
\usepackage[utf8]{inputenc}
\usepackage{float}
\usepackage{amsmath}


\usepackage[hmargin=3cm,vmargin=6.0cm]{geometry}
\topmargin=-2cm
\addtolength{\textheight}{6.5cm}
\addtolength{\textwidth}{2.0cm}
\setlength{\oddsidemargin}{0.0cm}
\setlength{\evensidemargin}{0.0cm}
\usepackage{indentfirst}
\usepackage{amsfonts}

\begin{document}

\section*{Student Information}

Name : Mustafa Sezgin \\

ID : 2380863 \\


\section*{Answer 1}

\subsection*{a)}
Let $\mu_{B}$, $\mu_{Y}$, and $\mu_{R}$ be the expected values of blue, yellow, and red dice respectively.
\begin{align*}
    \mu_{B} &= E(B) = 2 \cdot \frac{4}{6} + 3 \cdot \frac{1}{6} + 4 \cdot \frac{1}{6} = \frac{15}{6} = 2.5 \\
    \mu_{Y} &= E(Y) = 1 \cdot \frac{2}{6} + 2 \cdot \frac{2}{6} + 3 \cdot \frac{2}{6} = \frac{12}{6} = 2 \\
    \mu_{R} &= E(R) = 1 \cdot \frac{2}{8} + 2 \cdot \frac{2}{8} + 3 \cdot \frac{3}{8} + 5 \cdot \frac{1}{8} = \frac{20}{8} = 2.5
\end{align*}

\subsection*{b)}
Expected value of rolling 2 red and 1 yellow dice is
\[ E(2R + Y) = 2 E(R) + E(Y) = 2 (2.5) + 2 = 7, \]
and similarly, expected value of rolling 2 yellow and 1 blue dice is
\[ E(2Y + B) = 2 E(Y) + E(B) = 2 (2) + 2.5 = 6.5, \]
by linearity of expectation. Thus, rolling 2 red and 1 yellow dice would be a better choice.

\subsection*{c)}
In this case, the expected value of rolling a blue die is $E(R) = 4 \cdot 1 = 4$. While nothing changes for the first option, the expected value of the second one will be
\[ E(2Y + B) = 2 E(Y) + E(B) = 2 (2) + 4 = 8. \]
Now, it would be better to choose the second option.

\subsection*{d)}
Let $B$, $Y$, $R$ denote the events choosing the blue, yellow, and red dice respectively, and $V$ denote the event "the value is 3". We need to find the probability of $C$ given $V$, that is, $P(C \mid V)$. By the \textit{Bayes Rule},
\[ P(C \mid V) = \frac{P(V \mid C) P(C)}{P(V)}, \]
where $P(V \mid C) = \cfrac{3}{8}$ is the probability of getting the value 3 when red die is rolled, and
\begin{align*}
    P(V) &= P(B \cap V) + P(Y \cap V) + P(R \cap V) \\
         &= P(V \mid B) P(B) + P(V \mid Y) P(Y) + P(V \mid R) P(R) \\
         &= \frac{1}{6} \cdot \frac{1}{3} + \frac{2}{6} \cdot \frac{1}{3} + \frac{3}{8} \cdot \frac{1}{3} \\
         &= \frac{7}{24},
\end{align*}
since the intersections are exhaustive and mutually exclusive. Thus,
\begin{align*}
    P(C \mid V) &= \frac{\cfrac{3}{8} \cdot \cfrac{1}{3}}{\cfrac{7}{24}} \\
    &= \frac{3}{7}.
\end{align*}

\subsection*{e)}
Let $V_1$, $V_3$, $V_5$ denote the events "the value is 1, 3, and 5" respectively. The total value will be 6 when $R = 3$ and $Y = 3$, or $R = 5$ and $Y = 1$, which are independent. The probability is
\begin{align*}
    P(V_3 \mid R) P(V_3 \mid Y) + P(V_5 \mid R) P(V_1 \mid Y) &= \frac{3}{8} \cdot \frac{2}{6} + \frac{1}{8} \cdot \frac{2}{6} \\
    &= \frac{1}{6}.
\end{align*}


\section*{Answer 2}

\subsection*{a)}
$P(A = 0, I = 2) = 0.17$, by the table.

\subsection*{b)}
$P(A = 2, I = 0) = 0$, since the table has no $A = 2$ data.

\subsection*{c)}
$P(A = 0, I = 2) + P(A = 1, I = 1) = 0.17 + 0.11 = 0.28$, since they are mutually exclusive. 
% \\ $P_{(A, I)}(0, 2) + P_{(A, I)}(1, 1) = 0.17 + 0.11 = 0.28$, by the table.

\subsection*{d)}
All rows are exhaustive and mutually exclusive. By addition rule,
\begin{align*}
    P_A(1) &= \sum_{i} P_{(A, I)}(1, i) \\
             &= P_{(A, I)}(1, 0) + P_{(A, I)}(1, 1) + P_{(A, I)}(1, 2) + P_{(A, I)}(1, 3) \\
             &= 0.12 + 0.11 + 0.22 + 0.15 \\
             &= 0.6
\end{align*}

\subsection*{e)}
Let $X = A + I$ be the total number of electric outages.
\begin{align*}
    P_X(0) &= P_{(A, I)}(0, 0) = 0.08 \\
    P_X(1) &= P_{(A, I)}(0, 1) + P_{(A, I)}(1, 0) = 0.25 \\
    P_X(2) &= P_{(A, I)}(0, 2) + P_{(A, I)}(1, 1) = 0.28 \\
    P_X(3) &= P_{(A, I)}(0, 3) + P_{(A, I)}(1, 2) = 0.24 \\
    P_X(4) &= P_{(A, I)}(1, 3) = 0.15
\end{align*}
Then, the distribution of $X$ is as follows.
\begin{center}
\begin{tabular}{c|c}
    $x$ & $P_X(x)$ \\
    \hline
    0 & 0.08 \\
    1 & 0.25 \\
    2 & 0.28 \\
    3 & 0.24 \\
    4 & 0.15
\end{tabular}
\end{center}

\subsection*{f)}
Assume they are independent. Then, for every $a$ and $i$, we should have $P(A) = P(A \mid I)$ and $P(I) = P(I \mid A)$. However, this does not hold when, for example, $A = 1$ and $I = 1$.
\begin{align*}
    P(A = 1 \mid I = 1) &= \frac{P_{(A, I)}(1, 1)}{P_{I}(1)} \\
    &= \frac{P_{(A, I)}(1, 1)}{P_{(A, I)}(0, 1) + P_{(A, I)}(1, 1)} \\
    &= \frac{0.11}{0.13 + 0.11} \\
    &= \frac{0.11}{0.24} \\
    &= 0.458,
\end{align*}
which is not equal to $P_{A}(1) = 0.6$ as calculated in part d. This contradicts with our assumption; thus, they are not independent.

\end{document}
