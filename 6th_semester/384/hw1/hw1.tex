\documentclass[10pt,a4paper, margin=1in]{article}
\usepackage{fullpage}
\usepackage{amsfonts, amsmath, pifont}
\usepackage{amsthm}
\usepackage{graphicx}
\usepackage{float}

\usepackage{tkz-euclide}
\usepackage{tikz}
\usepackage{pgfplots}
\pgfplotsset{compat=1.13}

\usepackage{geometry}
 \geometry{
 a4paper,
 total={210mm,297mm},
 left=10mm,
 right=10mm,
 top=10mm,
 bottom=10mm,
 }
 % Write both of your names here. Fill exxxxxxx with your ceng mail address.
 \author{
  LastName1, FirstName1\\
  \texttt{exxxxxxx@ceng.metu.edu.tr}
  \and
  LastName2, FirstName2\\
  \texttt{exxxxxxx@ceng.metu.edu.tr}
}

\title{CENG 384 - Signals and Systems for Computer Engineers \\
Spring 2022 \\
Homework 1}
\begin{document}
\maketitle



\noindent\rule{19cm}{1.2pt}

\begin{enumerate}

\item %write the solution of q1
    \begin{enumerate}
    % Write your solutions in the following items.
    \item %write the solution of q1a
    \item %write the solution of q1b
    \item %write the solution of q1c
    \item %write the solution of q1d
    \end{enumerate}

\item %write the solution of q2
    \begin{enumerate}
    % Write your solutions in the following items.
    \item %write the solution of q2a
    \item %write the solution of q2b
    \end{enumerate}

\item %write the solution of q3  

\item %write the solution of q4
    \begin{enumerate}
    % Write your solutions in the following items.
    \item %write the solution of q4a
    \item %write the solution of q4b
    \end{enumerate}

\item %write the solution of q5
    \begin{enumerate}   
    % Write your solutions in the following items.
    \item %write the solution of q5a
    \item %write the solution of q5b
    \end{enumerate}

\item %write the solution of q6
    \begin{enumerate}
    % Write your solutions in the following items.
    \item %write the solution of q6a
    \item %write the solution of q6b
    \end{enumerate}    
    
\item %write the solution of q7
    \begin{enumerate}
    % Write your solutions in the following items.
    \item %write the solution of q7a
    \item %write the solution of q7b
    \end{enumerate}
    
\item %write the solution of q8
    \begin{enumerate}
    % Write your solutions in the following items.
    \item %write the solution of q8a
    \item %write the solution of q8b
    \end{enumerate}    

\end{enumerate}

% \begin{figure}[h!]
%     \centering
%         \begin{tikzpicture}[scale=1.0]
%           \begin{axis}[
%           axis lines=middle,
%           xlabel={$t$},
%           ylabel={$\boldsymbol{x(t)}$},
%           xtick={-4, -3, -2, -1, ..., 4},
%           ytick={-3, -2, -1, ..., 3},
%           ymin=-3, ymax=3,
%           xmin=-4, xmax=4,
%           every axis x label/.style={at={(ticklabel* cs:1.05)}, anchor=north,},
%           every axis y label/.style={at={(ticklabel* cs:1.05)}, anchor=south,},
%           grid,
%         ]
%           \path[draw,line width=4pt] (-4,0) -- (-3,0) -- (-2,0) -- (-1,1) -- (1,1) -- (1,0) -- (3,0) -- (4,0);
%           \end{axis}
%         \end{tikzpicture}
%         \caption{$t$ vs. $x(t)$.}
%         \label{fig:fig1}
%     \end{figure}
    
% \begin{filecontents}{fig2.dat}
%  n   xn
%  -1  1
%  0   0
%  1   0  
%  2   -2
%  3   0
%  4   3 
%  5   0
%  6   0
%  7   -4
% \end{filecontents}

% \begin{figure}[h!]
%     \centering
%     \begin{tikzpicture}[scale=1.0] 
%       \begin{axis}[
%           axis lines=middle,
%           xlabel={$n$},
%           ylabel={$\boldsymbol{x[n]}$},
%           xtick={ -1, 0,  ..., 7},
%           ytick={-4, -3, -2, -1, ..., 4},
%           ymin=-4, ymax=4,
%           xmin=-1, xmax=7,
%           every axis x label/.style={at={(ticklabel* cs:1.05)}, anchor=west,},
%           every axis y label/.style={at={(ticklabel* cs:1.05)}, anchor=south,},
%           grid,
%         ]
%         \addplot [ycomb, black, thick, mark=*] table [x={n}, y={xn}] {fig2.dat};
%       \end{axis}
%     \end{tikzpicture}
%     \caption{$n$ vs. $x[n]$.}
%     \label{fig:fig2}
% \end{figure}

\end{document}

